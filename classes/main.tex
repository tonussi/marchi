\documentclass[12pt, a4paper]{article}

\usepackage[utf8]{inputenc}
\usepackage[brazil]{babel}

\usepackage[section]{placeins}

\usepackage[lmargin=2cm, rmargin=2cm, tmargin=2cm, bmargin=2cm]{geometry}

\usepackage{graphicx, amsmath, amssymb, graphicx, enumerate, , hyperref, color, libertine, listings, sectsty, pdfcomment, parskip}

\newcommand{\HRule}{\rule{\linewidth}{1pt}}

\sectionfont{\LARGE}
\subsectionfont{\Large}
\subsubsectionfont{\large}

\newif\ifxetexorluatex
\ifxetex
  \xetexorluatextrue
\else
  \ifluatex
    \xetexorluatextrue
  \else
    \xetexorluatexfalse
  \fi
\fi

\ifxetexorluatex%
  \usepackage{fontspec}
  \usepackage{libertine} % or use \setmainfont to choose any font on your system
  \newfontfamily\quotefont[Ligatures=TeX]{Linux Libertine O} % selects Libertine as the quote font
\else
  \usepackage[utf8]{inputenc}
  \usepackage[T1]{fontenc}
  \usepackage{libertine} % or any other font package
  \newcommand*\quotefont{\fontfamily{LinuxLibertineT-LF}} % selects Libertine as the quote font
\fi

\begin{document}

\begin{titlepage}
\begin{center}

\textsc{\large Universidade Federal de Santa Catarina}\\[1cm]

\includegraphics[width=0.3\textwidth]{ufsc}\\[1.5cm]

\textsc{\LARGE \bfseries Teoria da Computação \\ [0.8cm]}

\textsc{\LARGE \bfseries Problema da Correspondência de Post \\ [3cm]}


\begin{Large}
\textbf{Estudante}:
\begin{tabular}{|c}
Luc$\lambda$s Tonussi 12106577\\
\end{tabular} \\[0.5cm]
\end{Large}

\vfill

\begin{Large}
\textbf{Professora}:
\begin{tabular}{|c}
Jerusa Marchi \\
\end{tabular} \\[0.5cm]
\end{Large}

\vfill


\textbf{\today}

\end{center}
\end{titlepage}

\tableofcontents
\pagebreak

\listoffigures
\pagebreak

Conforme visto em sala, um problema \textbf{NP-Completo} é um problema que

\begin{enumerate}
  \item está em \textbf{NP}.
  \item todo problema $\Pi \in \text{\textbf{NP}}$ pode ser reduzido a ele em tempo polinomial.
\end{enumerate}

Também vimos que o problema da \textbf{Satisfazibilidade Booleana} (SAT) foi o primeiro problema a ser demonstrado \textbf{NP-Completo}. Há na literatura uma série de outros problemas provados \textbf{NP-Completos} \cite{newton04}.

Em geral, a prova desta asserção consiste na redução de um problema reconhecidamente \textbf{NP-Completo} ao que se quer demonstrar.

O trabalho 3 consiste na pesquisa e apresentação da redução de um problema \textbf{NP-Completo}
a outro. Alguns problemas clássicos são listados abaixo. Outros problemas podem ser encontrados \href{http://en.wikipedia.org/wiki/List_of_NP-complete_problems}{aqui!}.

\begin{itemize}
  \item Caminho Hamiltoniano.
  \item Caixeiro Viajante.
  \item Caminho mais longo.
  \item Clique.
  \item Mochila.
  \item Cobertura Exata.
  \item Vértices de Cobertura.
  \item Conjuntos Independentes.
  \item Roteamento de veículos.
  \item Isomorfismo em grafos.
  \item Caminho Rudrata.
  \item Corte Balanceado.
  \item Soma de Subconjuntos.
\end{itemize}

Para tanto pede-se:

\begin{itemize}
  \item Enuncie os problemas envolvidos na redução.
  \item Esclareça o sentido da redução (problema fonte, problema destino).
  \item Apresente a redução.
\end{itemize}

\pagebreak
\begin{thebibliography}{9}
\bibitem {newton04} Vieira J, N. Linguagens e Máquinas: Uma Introdução aos Fundamentos da Computação, Departamento de Ciência da Computação - Instituto de Ciências Exatas Universidade Federal de Minas Gerais - Belo Horizonte, 2004.

\bibitem {sipser06} Sipser, M. Introduction to the Theory of Computation (2nd Ed), Thomson Course Technology © 2006, ISBN-13: 978-0-534-95097-2, ISBN-10: 0-534-95097-3.
\end{thebibliography}

\end{document}
